\documentclass[%
 reprint,
 superscriptaddress,
 amsmath,
 amssymb,
 prl,
]{revtex4-1}

\usepackage{graphicx}% Include figure files
\usepackage{dcolumn}% Align table columns on decimal point
\usepackage{bm}% bold math
\usepackage{color}

\begin{document}

\title{Stabilisation of the Arrival Time of a Relativistic Electron Beam to 
the 50~fs Level}

\author{J.~Roberts}
\email{Corresponding author Jack.Roberts@cern.ch}
\affiliation{John Adams Institute (JAI), University of Oxford, Denys Wilkinson 
Building, Keble Road, Oxford, OX1 3RH, United Kingdom}
\affiliation{The European Organization for Nuclear Research (CERN), Geneva 23, 
CH-1211, Switzerland}

\author{P.~Skowronski}
\affiliation{The European Organization for Nuclear Research (CERN), Geneva 23, 
	CH-1211, Switzerland}

\author{P.N.~Burrows}
\affiliation{John Adams Institute (JAI), University of Oxford, Denys Wilkinson 
Building, Keble Road, Oxford, OX1 3RH, United Kingdom}

\author{G.B.~Christian}
\affiliation{John Adams Institute (JAI), University of Oxford, Denys Wilkinson 
Building, Keble Road, Oxford, OX1 3RH, United Kingdom}

\author{R.~Corsini}
\affiliation{The European Organization for Nuclear Research (CERN), Geneva 23, 
CH-1211, Switzerland}

\author{A.~Ghigo}
\affiliation{Laboratori Nazionali di Frascati (LNFN), Via Enrico Fermi, 40, 
00044 Frascati RM, Italy}

\author{F.~Marcellini}
\affiliation{Laboratori Nazionali di Frascati (LNFN), Via Enrico Fermi, 40, 
00044 Frascati RM, Italy}

\author{C.~Perry}
\affiliation{John Adams Institute (JAI), University of Oxford, Denys Wilkinson 
Building, Keble Road, Oxford, OX1 3RH, United Kingdom}


\date{\today}

\begin{abstract}
We report the results of a low-latency beam phase feed-forward system built to 
stabilise the arrival time of a relativistic electron beam. The system was 
operated at the Compact Linear Collider (CLIC) Test Facility (CTF3) at CERN 
where the beam arrival time was stabilised to approx. 50~fs. The system 
latency was \(350\)~ns and the correction bandwidth \(>23\)~MHz. 
The system meets the requirements for CLIC and could have applications at 
future free-electron lasers.
\end{abstract}

\maketitle

%%

%High-energy linear electron-positron colliders have been proposed as 
%next-generation particle accelerators for exploring the subatomic world with 
%increased precision. They will provide sensitivity to new physics processes, 
%beyond those described by the Standard Model (SM) of elementary particle 
%interactions, at mass scales that can exceed the eventual reach of the CERN 
%Large Hadron Collider (LHC) by more than an order of magnitude. 

%The Compact Linear Collider (CLIC) has been proposed~\cite{CLICCDR} as a 
%particle physics facility for the annihilation of electrons and positrons at 
%centre-of-mass energies of up to 3 TeV. CLIC is the most technologically 
%mature 
%concept of a high-energy lepton collider for enabling direct searches for new 
%physics processes in the multi-TeV energy regime. This energy reach, combined 
%with high-luminosity of the electron-positron collisions, will also enable 
%precise measurements of properties of the Higgs boson~\cite{CLIC-Higgs} and 
%the 
%top quark, and provide sensitivity to beyond-SM phenomena at mass scales of up 
%to 10-100~TeV in some cases~\cite{CLIC-staging}.

High-energy linear electron-positron colliders have been proposed as 
next-generation particle accelerators for exploring the subatomic world with 
increased precision. They provide sensitivity to new physics processes, 
beyond those described by the Standard Model (SM) of elementary particle 
interactions, at mass scales that can exceed the reach of the CERN 
Large Hadron Collider (LHC) by more than an order of magnitude 
\cite{CLIC-staging}.

The Compact Linear Collider (CLIC)~\cite{CLICCDR} is the most technologically 
mature concept of a high-energy lepton collider for enabling direct searches 
for new physics in the multi-TeV energy regime. It uses a novel two 
beam acceleration concept to achieve a high accelerating gradient of 100 MV/m 
and centre-of-mass collision energies of up to 3~TeV. This energy reach, 
combined with high-luminosity of the electron-positron collisions, will also 
enable precise measurements of properties of the Higgs boson~\cite{CLIC-Higgs} 
and the top quark, and provide sensitivity to beyond-SM phenomena at mass 
scales of up to 10-100~TeV~\cite{CLIC-staging}.

The CLIC two beam acceleration concept is shown schematically in 
Fig.~\ref{fig:CLICLayout}. The 12~GHz RF power used to accelerate the main high 
energy electron and positron beams is generated by decelerating high intensity 
electron `drive beams'. The power is extracted from the drive beams and 
transferred to the main beams in a number of decelerator sectors; two drive 
beams with 25 decelerator sectors each are required for a 3~TeV collider. Prior 
to power extraction the drive beam comprises a 240~ns-long pulse of 2.4~GeV 
electrons bunched with a frequency of 12~GHz; the pulse repetition rate is 
50~Hz.

%CLIC employs the novel concept of high power generation at radio-frequency 
%%%(RF) by decelerating an electron ‘drive beam’ and utilising 
%that power to accelerate the main electron and positron beams to 
%high energies. The drive-beam concept is shown schematically in 
%Fig.~\ref{fig:CLICLayout}; 50 deceleration sections are required for a 3 TeV 
%electron-positron collider. At the decelerators the drive beam comprises a 
%240~ns-long pulse of 2.4~GeV electrons bunched with a frequency of c. 12~GHz; 
%the pulse repetition rate is 50~Hz.

A major challenge is the synchronisation of the arrival of the drive and main 
beams at the power-extraction and transfer structures to better than 50~fs rms. 
This requirement limits the luminosity loss, resulting from subsequent 
energy errors of the main beams, to less than 1\% of the design 
value~\cite{clicLumEq}. X-ray free-electron lasers also demand a high degree 
of beam arrival-time stability w.r.t. an externally-applied laser beam for the 
purpose of seeding of lasing by the electron beam. An RF phase and amplitude 
feedback utilising electro-optic beam arrival time monitors in this context was 
reported in~\cite{flashPRL}.

\begin{figure}
	\includegraphics[width=\columnwidth]{figs/clicLayout}
	\caption{\label{fig:CLICLayout} Schematic of the CLIC drive-beam 
	concept showing the electron acceleration complex \cite{CLIC-staging}.
	}
\end{figure}

We express the temporal stability of the drive beam in terms of a RF phase 
stability at the 12~GHz acceleration frequency. An arrival time jitter of 50~fs 
is equivalent to a phase jitter of \(0.2^\circ\)~at~12~GHz. `Jitter' refers to 
the rms of the quantity.
In the CLIC design the incoming drive-beam phase jitter 
cannot be guaranteed to be better than \(2^\circ\)~\cite{CLICCDR}. A mechanism 
to improve the phase stability by an order of magnitude is 
therefore required. The correction must be applied to the full drive beam pulse 
length and have a bandwidth exceeding 17.5~MHz~\cite{Gerber2015}. 

This is implemented via a `phase feed-forward' (PFF) system which measures the 
incoming beam phase and provides a correction to the same beam pulse 
after it has traversed the turnaround loop (TA in Fig.~\ref{fig:CLICLayout}). 
One PFF system will be installed in each deceleration section. The correction 
is provided by electromagnetic kickers in a 4-bend chicane: bunches arriving 
early (late) in time have their path through the chicane lengthened (shortened) 
respectively. A particular challenge is that the PFF latency must be shorter 
than the beam flight time of approx.~250~ns around the turnaround loop.

We describe a prototype PFF system (Fig.~\ref{fig:pffLayout}) that implements 
this novel concept at the CLIC Test Facility (CTF3) at CERN. CTF3 provides a 
135~MeV electron beam bunched at 3~GHz frequency with a beam-pulse length of 
1.2~\(\mathrm{\mu s}\) and a pulse repetition rate of 0.8~Hz \cite{CLICCDR}. 

The incoming beam phase is measured in two upstream phase 
monitors (\(\phi_{1}, \phi_{2}\)). While the beam 
transits the ‘turnaround loop’ a phase-correction signal is evaluated and used 
to drive fast, high power amplifiers; these drive two electromagnetic kickers 
(\(\mathrm{K_1, K_2}\)) which are used to alter the beam transit time in a 
four-bend chicane. A downstream phase monitor (\(\phi_{3}\)) is 
used to measure the effect of the correction. 

The beam time of flight between \(\phi_1\) and \(\mathrm{K_1}\) is around 
380~ns. The total cable delay for the PFF correction signals 
is shorter, around 250~ns. The correction in the chicane can therefore be 
applied to entirety of the beam pulse measured at \(\phi_1\), the PFF input, 
providing the hardware latency is less than 130~ns. Significant hardware 
challenges include the resolution and bandwidth of the phase monitors, and the 
power, latency and bandwidth of the kicker amplifiers. A low latency 
digitiser/feedforward controller is also required.
 
\begin{figure}
	\includegraphics[width=\columnwidth]{figs/ctfpffLayout}% Here is 
	%how to 
	%import EPS art
	\caption{\label{fig:pffLayout}Schematic of the CTF3 PFF prototype, 
	showing the phase monitors (\(\phi_1\) , 
	\(\phi_2\) and \(\phi_3\)) and kickers (K1 and K2). The black box “PFF” 
	represents the calculation and output of the correction, including the 
	phase monitor signal processing electronics, feedforward controller and 
	kicker amplifiers. Dashed lines indicate beam lines that are not used. 
		}
\end{figure}

The requirements of the CLIC system and their corresponding CTF3 values are 
listed in Table~\ref{tab:pffspecs}. The main differences result from the 
different drive beam energies. Higher power 
amplifiers (500~kW rather than 20~kW) are required for CLIC, which may be 
achieved by combining the output of multiple modules similar to those built for 
CTF3.
CLIC also requires a distributed timing system to synchronise the phase of the 
drive and main beams along the 50~km facility, which is not addressed here.

\begin{table}
	\caption{\label{tab:pffspecs}
	    Requirements for the CLIC PFF system, and the respective CTF3 
	    parameters; performance achieved with the prototype system is indicated 
	    by \textbf{*}.}
\begin{ruledtabular}
	\begin{tabular}{lccc}
		 & CLIC & CTF3 \\
		\hline
		Drive beam energy & 2400 & 135 & MeV \\
		No. PFF systems & 50 & 1 & \\
		Kickers per PFF chicane & 16 & 2 & \\
		Power of kicker amplifiers & 500 & \(\mathbf{20^*}\) & kW \\
		Angular deflection per kicker & \(\pm94\) & 
		\(\mathbf{\pm560^*}\) & \(~\mathrm{\mu rad}\) \\
		Correction range & \(\pm 10\) & \(\mathbf{\pm 6^*}\) & \(^\circ\) \\
		Correction bandwidth & \(>17.5\) & \(\mathbf{>23^*}\) & MHz \\
		Phase monitor resolution & \(< 0.14\) & \(\mathbf{0.12^*}\) &  
		\(^\circ\)   \\
		Initial phase jitter & \(2.0\) & \(0.9\) &  \(^\circ\) \\
		Corrected phase jitter & \(0.2\) & \(\mathbf{0.2^*}\) &  \(^\circ\)  \\
	\end{tabular}
\end{ruledtabular}
\end{table}


The phase monitors~\cite{phMonEuCard} are cylindrical cavities with an aperture 
of 23~mm and a length of 19~cm. Small ridges (’notch filters’) in the cavity 
create an effective volume with a resonant frequency of 12~GHz. 
The resonant electromagnetic field induced by the beam traversing the cavity 
contains a beam-position-independent monopole mode and an unwanted 
position-dependent dipole mode. The 
effect of the latter is removed by summing the outputs from an opposing pair 
of feedthroughs, on the top and bottom of the cavity, via a RF ‘hybrid’. 
To extract the beam phase the output from each hybrid 
is mixed with a 12~GHz reference signal derived from a 3~GHz source which is 
phase-locked to the CTF3 RF system and serves all three phase monitors.
A linear response to input beam phase was measured over the range 
\(\pm70^\circ\) \cite{Skowron2013}. By comparing the signals from 
\(\phi_1\)~and~\(\phi_2\) we have measured a phase resolution of 
\(0.12^\circ\), i.e. about 30~fs~\cite{RobertsThesis}.

The phase signals are digitised  in the feedforward controller
board~\cite{RobertsThesis}, which is used to calculate and output the amplifier 
drive signals, and to  control the correction timing. It consists of nine 
14-bit analogue to digital converters clocked at 357~MHz, a field programmable 
gate array, and four digital to analogue converters. 

The kicker amplifiers~\cite{RobertsThesis} consist of one central control 
module and two drive and terminator modules (one per kicker). The control 
module distributes power and input signals to the 
drive modules. The 20~kW drive modules consist of low-voltage Si FETs driving 
high-voltage SiC FETs; an input voltage range of \(\pm2\)~V corresponds to an 
output range of \(\pm700\)~V. The response is linear to within 3\% for input 
voltages between \(\pm1.2\)~V, and the output bandwidth is 47~MHz for small 
signal variations of up to 20\% of the maximum. For larger signal variations 
the bandwidth is slew-rate limited.

The two electromagnetic stripline kickers are 1~m in length 
and have an internal aperture of 40~mm between two strips placed along their 
horizontal walls. They are designed to give a response within a few ns of the 
input signal. The strips have tapered ends to reduce beam coupling 
impedance~\cite{kickerIPAC11}. 
Opposite polarity voltages of up to 700~V applied to the 
strips at the downstream end produce a horizontal deflection of up to 
560~\(\mu\)rad for the 135~MeV beam.

The measured total latency of the phase monitor signal processing, the 
feedforward calculation, and amplifier response was approximately 100~ns. 
Therefore the output from the controller was delayed by an additional 30~ns to 
synchronise the correction at the kicker with the beam arrival 
\cite{RobertsThesis}.

\begin{figure}
	\includegraphics[width=\columnwidth]{figs/corrRange}
	\caption{\label{fig:corrRange}Measured downstream beam phase vs. kicker 
	amplifier input voltage. Standard errors are shown.}
\end{figure}

The PFF operation placed severe constraints on the setting of the 
magnetic lattice in both the beamline between the upstream phase monitors and 
the correction chicane, and in  the chicane itself.
The beam transfer matrix coefficient \(R_{52}\) between the two kickers 
characterises the change in path length through the chicane relative to the 
deflection applied at the first kicker. 
With an \(R_{52}\) value of \(0.74\)~m/rad \cite{RobertsThesis} the expected 
maximum path length change for operation of the PFF system, corresponding to 
the maximum deflection of \(\pm560\)~\(\mu\)rad from each kicker, is about 
\(\pm400~\mathrm{\mu m}\), equivalent to \(\pm6^\circ\) in phase 
(Fig.~\ref{fig:corrRange}). PFF operation also should not change the beam 
trajectory at the exit of the chicane. Therefore the chicane magnets were set 
so that the second kicker cancels the transverse orbit deviation created by the 
first~\cite{RobertsThesis}.

A further challenge to PFF operation was obtaining a high correlation 
between the upstream and uncorrected downstream phases measured at \(\phi_1\) 
and \(\phi_3\) respectively. 
A correlation coefficient of at least 97\% is required to reduce a typical 
incoming phase jitter of \(0.8^\circ\) to the target of \(0.2^\circ\) 
\cite{RobertsThesis}. 
The maximum measurable correlation depends on both the phase monitor resolution 
and any additional phase jitter introduced in the beamlines between \(\phi_1\) 
and \(\phi_3\). The monitor resolution of \(0.12^\circ\) limits the maximum 
upstream-downstream phase correlation to \(98\%\) in typical conditions, and 
places a theoretical limit of \(0.17^\circ\) on the measurable corrected 
downstream phase jitter. 
Alternatively, the dominant beam source of uncorrelated downstream phase jitter 
arises from energy jitter that is transformed into phase jitter in the 
beamlines between \(\phi_1\) and \(\phi_3\). 

\begin{figure}
	\includegraphics[width=\columnwidth]{figs/r56Scan}
	\caption{\label{fig:r56Scan}Measured downstream (red) and upstream (blue) 
	phase jitter vs. TL1 \(R_{56}\) value. Standard errors are shown.
		}
\end{figure}

To first order the phase-energy dependence can be described via the beam 
transfer matrix coefficient 
\(R_{56}\): \(\phi_3 = \phi_1 + R_{56}(\Delta p / p)\)
, where \(\Delta p / p\) is the particle's relative energy error.
The optimal condition is \(R_{56}\) = 0.
This was achieved by tuning the \(R_{56}\) value in the `TL1' transfer line 
(Fig.~\ref{fig:pffLayout}) so as to compensate for non-zero \(R_{56}\) in the 
other beamline sections. With \(R_{56, \mathrm{TL1}}=10\)~cm the 
downstream phase jitter is reduced to the same level as the upstream jitter 
(Fig.~\ref{fig:r56Scan}). 
However, a large second-order phase-energy dependence remained uncorrected, 
resulting in a degradation in upstream-downstream phase correlation for large 
drifts in beam energy.

%\begin{equation*}
%\sigma_{\mathrm{PFF}}^2 = \sigma_d^2 + g^2\sigma_u^2 - 
%2g\rho_{ud}\sigma_u\sigma_d
%\end{equation*}
The PFF system acts to remove a multiple of the \(\phi_1\) phase from the phase 
at \(\phi_3\). If the phases at \(\phi_3\) and \(\phi_1\) are fully 
correlated, 
and the jitters are identical, the optimal multiplication factor, or `gain',  
is unity.
In practice the gain is chosen to achieve optimal 
performance for real beam conditions. A representative gain scan is shown 
in~Fig.~\ref{fig:gScan}. The optimal system gain is typically in the range 
0.9--1.3. Also shown in Fig.~\ref{fig:gScan} is a theoretical prediction of the 
corrected phase jitter given the initial beam phase jitter at \(\phi_1\) and 
\(\phi_3\), the upstream-downstream phase correlation, and the gain. The 
simulation reproduces the data.

\begin{figure}
\includegraphics[width=\columnwidth]{figs/gScan}
\caption{\label{fig:gScan}Measured corrected beam  phase jitter at \(\phi_3\) 
vs. PFF gain; standard errors are shown (points). The theoretically-achieveable 
performance is shown by the red shaded region (see text).}
\end{figure}

The PFF system simultaneously corrects pulse-to-pulse phase jitter and phase 
variations within the 1.2~\(\mu s\) beam pulse at CTF3. 
Fig.~\ref{fig:shape} shows the effect of the PFF system on the intra-pulse 
phase variations. The PFF system was operated in interleaved mode, with 
the correction applied to alternating pulses only. This allows 
the initial (`PFF Off') and corrected (`PFF On') downstream phase 
to be measured concurrently at \(\phi_3\). The \(\phi_1\) (PFF input) phase 
is also shown for comparison. 

It is an operational feature at CTF3 that there is a roughly parabolic phase 
sag of \(40^\circ\), resulting from the upstream RF pulse compression 
scheme~\cite{CLICCDR}. Hence approximately a 440~ns portion of the pulse is 
within the \(\pm 6^\circ\) dynamic range of the PFF system, and can be 
corrected to zero nominal phase. 
This time duration for the full correction exceeds the CLIC drive-beam pulse 
length of 240ns and in any case the CLIC design avoids such 
a large phase sag~\cite{CLICCDR}. 
Vertical dashed lines in Fig.~\ref{fig:shape} mark the 440~ns portion of 
the pulse where full correction is possible, and this range is used in the 
following analyses. 

\begin{figure}
	\includegraphics[width=\columnwidth]{figs/shape}
	\caption{\label{fig:shape}Phase vs. time within the central portion of the  
	beam pulse; the incoming phase measured in \(\phi_1\) 
	(green), and the downstream phase measured in \(\phi_3\) with PFF off 
	(blue) and PFF on (red). Each trace is the average over a 30 minute dataset.
	The vertical dashed lines mark the time interval corresponding to the PFF 
	dynamic range. }
\end{figure}

Within the range the PFF system flattens the phase, and almost all variations 
are removed. 
Residual offsets are still present where there are small uncorrelated 
differences between the initial phase at \(\phi_1\) and \(\phi_3\). 
The average intra-pulse phase variation (rms) over the dataset is reduced from 
\(0.960\pm0.003^\circ\) (PFF off), to \(0.285\pm0.004^\circ\) (PFF on).

\begin{figure}
	\includegraphics[width=\columnwidth]{figs/fft}
	\caption{\label{fig:fft}Amplitude of phase errors at different frequencies 
		(\(f\)) with the PFF system off (blue) and on (red).}
\end{figure}

In order to meet CLIC requirements (Table~\ref{tab:pffspecs}) the PFF 
correction bandwidth should be at least 17.5 MHz. 
A Fourier-Transform (FFT) method was used to characterise the PFF on/off 
datasets. The FFT amplitude is shown vs. frequency in 
Fig.~\ref{fig:fft}. It can be seen that phase errors are corrected by up to a 
factor of 5 for frequencies up to 23~MHz, above which 
they are smaller than the monitor resolution and not measurable. This 
is consistent with an expected system bandwidth of around 30~MHz, and exceeds 
the CLIC requirement.

The effect of the PFF system on the pulse-to-pulse jitter, or the jitter on the 
mean phase of each beam pulse, is shown in Fig~\ref{fig:meanJit} for a dataset 
of around ten minutes duration.
The pulse-to-pulse phase jitter is reduced from  \(0.92\pm0.04^\circ\) to 
\(0.20\pm0.01^\circ\), meeting CLIC-level phase stability. 
The system acts to remove all correlation between the upstream and 
downstream phase, reducing an initial correlation of \(96\pm2\%\) to 
\(0\pm7\%\) for this dataset.
Given the incoming upstream phase jitter and 
measured upstream-downstream correlation, the performance is consistent with 
the theoretically predicted correction of \(0.26\pm0.06^\circ\).

\begin{figure}
	\includegraphics[width=\columnwidth]{figs/meanJit}
	\caption{\label{fig:meanJit}Distribution of the mean downstream phase with 
		the 
		PFF system off (blue) and on (red).}
\end{figure}

Typically this level of corrected phase stability could not be maintained for 
longer time periods due to drifts in the CTF3 RF system, which 
led to  a degradation in the upstream-downstream phase correlation as well as 
mean phase drifts beyond the PFF correction range. Nevertheless a mean phase 
stability of \(0.30^\circ\) was achieved in datasets taken over periods as long 
as 20~minutes. With suitable upstream RF feedbacks to keep the beam phase 
within the correction range, and a reduction of the higher order phase-energy 
dependence in the magnetic lattice, the system is capable of achieving 
CLIC-level phase stability continuously.

The system was further tested by varying the incoming mean 
beam phase systematically by up to \(3^\circ\); variations of this magnitude 
are comparable to the expected conditions in the CLIC design 
(Table~\ref{tab:pffspecs}). This is illustrated in Fig.~\ref{fig:wiggle}. The 
system removed the induced phase variations and achieved more than a factor-5 
reduction in the downstream phase jitter, 
correcting~from~\(1.71\pm0.07^\circ\)~to~\(0.32\pm0.01^\circ\). 

\begin{figure}
	\includegraphics[width=\columnwidth]{figs/wiggle}
	\caption{\label{fig:wiggle}Mean downstream phase vs. time with the PFF 
	system off (blue) and on (red) subject to large additional phase variations 
	added to the incoming phase (see text).}
\end{figure}

%%%%%%%%%%%%

In conclusion, we have built, deployed and tested a prototype drive-beam phase 
feedforward system for CLIC.   The system incorporates purpose-built 
high-resolution phase 
monitors, an advanced signal-processor and feedforward controller, low-latency, 
high-power, high-bandwidth amplifiers, and electromagnetic stripline
kickers. The phase-monitor resolution was measured to be 
\(0.12^\circ\simeq\)~30~fs.  The overall system latency, including the hardware 
and signal transit times, was measured to be approx. 350~ns, which is less than 
beam time of flight between the input phase monitor and the correction 
chicane.  Therefore, the feedforward phase correction was applied downstream to 
the same beam bunches initially measured upstream. The system was used to 
stabilise the pulse-to-pulse phase jitter to \(0.20\pm0.01^\circ\simeq\)~50 fs, 
and simultaneously corrects intra-pulse phase variations at frequencies up to 
23~MHz. 

\begin{acknowledgments}
	We thank Alessandro Zolla and Giancarlo Sensolini (INFN 
	Frascati) for their work on the mechanical design of the phase monitors and 
	kickers, and Alexandra Andersson, Luca Timeo and Stephane Rey (CERN) for 
	their work on the phase monitor electronics. We thank the operations team 
	of CTF3 for their outstanding support. We acknowledge the UK Science and 
	Technology Facilities Council for their financial support for this work. 
	This work supported by the European Commission under the FP7 Research 
	Infrastructures project Eu-CARD, grant agreement no. 227579
\end{acknowledgments}

\bibliography{pff_prl}

\end{document}
