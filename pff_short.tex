% ****** Start of file apssamp.tex ******
%
%   This file is part of the APS files in the REVTeX 4.1 distribution.
%   Version 4.1r of REVTeX, August 2010
%
%   Copyright (c) 2009, 2010 The American Physical Society.
%
%   See the REVTeX 4 README file for restrictions and more information.
%
% TeX'ing this file requires that you have AMS-LaTeX 2.0 installed
% as well as the rest of the prerequisites for REVTeX 4.1
%
% See the REVTeX 4 README file
% It also requires running BibTeX. The commands are as follows:
%
%  1)  latex apssamp.tex
%  2)  bibtex apssamp
%  3)  latex apssamp.tex
%  4)  latex apssamp.tex
%
\documentclass[%
 reprint,
superscriptaddress,
%groupedaddress,
%unsortedaddress,
%runinaddress,
%frontmatterverbose, 
%preprint,
%showpacs,preprintnumbers,
%nofootinbib,
%nobibnotes,
%bibnotes,
 amsmath,amssymb,
 prl,
% aps,
%pra,
%prb,
%rmp,
%prstab,
%prstper,
%floatfix,
]{revtex4-1}

\usepackage{graphicx}% Include figure files
\usepackage{dcolumn}% Align table columns on decimal point
\usepackage{bm}% bold math
\usepackage{color}
\usepackage{verbatim}

%\usepackage{hyperref}% add hypertext capabilities
%\usepackage[mathlines]{lineno}% Enable numbering of text and display math
%\linenumbers\relax % Commence numbering lines

%\usepackage[showframe,%Uncomment any one of the following lines to test 
%%scale=0.7, marginratio={1:1, 2:3}, ignoreall,% default settings
%%text={7in,10in},centering,
%%margin=1.5in,
%%total={6.5in,8.75in}, top=1.2in, left=0.9in, includefoot,
%%height=10in,a5paper,hmargin={3cm,0.8in},
%]{geometry}

\begin{document}

%\preprint{APS/123-QED}

\title{High Bandwidth Arrival Time Stabilisation of an Electron Beam at 
the 
50~fs Level}
%\thanks{A footnote to the article title}%

\author{J.~Roberts}
\email{Corresponding author Jack.Roberts@cern.ch}
\affiliation{John Adams Institute,University of Oxford}
\affiliation{CERN, Geneva}
%\altaffiliation{CERN, Geneva.}

\author{P.~Burrows}
\affiliation{John Adams Institute,University of Oxford}

\author{G.~Christian}
\affiliation{John Adams Institute,University of Oxford}

\author{R.~Corsini}
\affiliation{CERN, Geneva}

\author{A.~Ghigo}
\affiliation{INFN/LNF, Frascati}

\author{F.~Marcellini}
\affiliation{INFN/LNF, Frascati}

\author{C.~Perry}
\affiliation{John Adams Institute,University of Oxford}

\author{P.~Skowronski}
\affiliation{CERN, Geneva}

\date{\today}

\begin{abstract}
CLIC, a proposed future linear electron-positron collider, and other machines 
such as XFELs, place tight tolerances on the phase stabilities of their beams. 
CLIC proposes the use of a novel, high bandwidth and low latency, `phase 
feedforward' system required to achieve a phase stability of 
\(0.2^\circ\)~at~12~GHz, or 
about 50~fs. This work documents the results from operation of a prototype 
phase 
feedforward system at the CLIC test facility CTF3, with \(>23\)~MHz bandwidth 
and a 
total hardware latency of 100~ns. New phase monitors with 
30~fs resolution, 20~kW amplifiers with 47~MHz bandwidth, and electromagnetic 
kickers have been designed and installed for the system. The system utilises a 
dog-leg chicane in the beamline, for which a dedicated optics have been created 
and 
commissioned. The prototype has demonstrated CLIC-level phase stability, 
reducing an initial rms phase variation of \(0.92\pm0.04^\circ\) to 
\(0.20\pm0.01^\circ\) across a duration of 10~minutes.
\end{abstract}

%\pacs{Valid PACS appear here}% PACS, the Physics and Astronomy
%                             % Classification Scheme.
%\keywords{Suggested keywords}%Use showkeys class option if keyword
                              %display desired
\maketitle

%\tableofcontents

%\section{\label{s:intro}Introduction}

The Compat Linear Collider, CLIC, \cite{CLICCDR} is a proposal for a future 
linear electron--positron collider. It uses a novel two beam acceleration 
concept to achieve a high accelerating gradient of 100~MV/m 
and a collision energy of up to 3 TeV. In this concept the 12~GHz RF power used 
to accelerate each high energy colliding beam is extracted and transferred from 
a high intensity drive beam in 24 decelerator sectors. \textcolor{red}{The 
drive beams are generated by compressing an initial 
\(140~\mathrm{\mu s}\) beam pulse bunched at 0.5~GHz into 24 shorter 240~ns 
beam pulses bunched at 12~GHz, in a bunch recombination process using a 
sequence 
of combiner rings and delay loops [REF]. }

CLIC's luminosity quickly drops if the drive beam phase, or arrival time, 
jitters with respect to the colliding beams, causing energy errors and 
subsequent beam size growth at the interaction point. The drive beam phase 
stability must be \(0.2^\circ\)~at~12~GHz (around 50~fs) rms or better to limit 
the luminosity loss to below 1\% \cite{CLICCDR}.  However, the drive beam phase 
stability cannot be guaranteed to be better than \(2^\circ\)~at~12~GHz [REF]. A 
mechanism to improve the drive beam phase stability by an order of magnitude is 
therefore required. The correction must be applied to the full drive beam pulse 
length and have a bandwidth exceeding 17.5~MHz to achieve this 
\cite{Gerber2015}. Higher frequency errors are filtered as a consequence of the 
drive beam recombination process, and by the accelerating structures 
\cite{Gerber2015}.

\textcolor{red}{Other machines, such as XFELs, have similar beam phase 
	stability 
	requirements to CLIC. At FLASH, DESY, these requirements have been met 
	using an RF phase 
	and power feedback based on the measurement of electro-optic beam arrival 
	time monitors [REF]. However, the CLIC drive beam presents a different set 
	of challenges. In particular, FLASH has 1~MHz bunch spacing and a 500~ms 
	beam pulse, whereas the CLIC drive beam has 12~GHz bunch spacing and 240~ns 
	pulse length. A feedback with a latency of several microseconds is 
	therefore not suitable for CLIC.}

CLIC instead proposes a drive beam ``phase feedforward'' (PFF) 
system. 
A prototype PFF system has been designed, commissioned and operated at 
the CLIC test facility CTF3, at CERN, to prove its feasibility. The prototype 
system follows the same concept as the CLIC scheme, and is the focus 
of this work. CTF3 provides a 135~MeV electron beam bunched at 3~GHz with a 
pulse length of 
1.2~\(\mathrm{\mu s}\) and a pulse repetition rate of 0.8~Hz [REF]. All phases 
quoted in the paper are given in degrees at 12~GHz, as relevant for CLIC.


%\section{\label{s:ctfLayout}System Design}

\begin{figure}
	\includegraphics[width=\columnwidth]{figs/ctfpffLayout}% Here is 
	%how to 
	%import EPS art
	\caption{\label{fig:pffLayout}Schematic of the PFF prototype at CTF3, 
	showing the approximate location of the phase monitors (\(\phi_1\) , 
	\(\phi_2\) and \(\phi_3\)) and
		the kickers (K1 and K2). The black box “PFF” represents the calculation 
		and output of the correction, including the phase monitor
		electronics, feedforward controller and kicker amplifiers. A bunch 
		arriving early at \(\phi_1\) is directed on to a longer path in the
		chicane
		using the kickers (blue trajectory), whereas a bunch arriving late will 
		be directed on to a shorter path (red trajectory). }
\end{figure}

A schematic of the prototype PFF system is shown in Fig.~\ref{fig:pffLayout}. 
The system corrects the phase using two electromagnetic kickers installed 
before the first and last dipole in a four bend, dog-leg shaped chicane. The 
beam's path length through the chicane depends on the magnitude and polarity of 
the voltage applied to the kickers. The phase is measured using a monitor 
upstream of the chicane, and then corrected by setting the kicker voltage to 
deflect bunches arriving early at the phase monitor on to longer trajectories 
in the chicane, and bunches arriving late on to shorter trajectories. 
Downstream of the chicane another phase monitor is placed to measure the 
effects of the correction.

The beam time of flight between the upstream phase monitor and the first kicker 
in the chicane is 380~ns. The layout of CTF3  means the total cable length 
required to transport signals between the monitor and kickers can be shorter, 
by bypassing some of the rings and transfer lines in the complex (see 
Fig.~\ref{fig:pffLayout}). The total cable length for the PFF correction 
contributes 250~ns to the system latency. The PFF correction in the chicane can 
therefore be applied to the same 
bunch initially measured at the phase monitor, providing the total system 
hardware latency is less than 130~ns. 

The PFF system presents a significant hardware challenge, in particular in 
terms of the power, latency and bandwidth requirements for the kicker 
amplifiers, and the resolution and bandwidth of the phase monitors. A low 
latency digitiser and feedforward controller is also required. New 
components have been designed and built for the prototype at CTF3, and a 
summary of their parameters in comparison to the CLIC requirements is shown in 
Table~\ref{tab:pffspecs}. 
The main differences between the CLIC and CTF3 systems result from the 
different drive beam energies (2.4~GeV at CLIC, 135~MeV at CTF3), and scales of 
the two facilities. Higher power amplifiers (500~kW rather than 20~kW) are 
required at CLIC, which may be achieved by combining the output of multiple 
modules similar to the CTF3 design [REF]. CLIC also requires the 
synchronisation of multiple PFF system distributed along the 50~km facility, 
which is not addressed by the CTF3 prototype (proposals can be found in [REF]).

\begin{table}
	\caption{\label{tab:pffspecs}
	    Overview of parameters and requirements for the prototype PFF system at 
	    CTF3, and how they compare to the proposed CLIC scheme.}
\begin{ruledtabular}
	\begin{tabular}{lcc}
		 & CLIC & CTF3 \\
		\hline
		No. Systems & 48 & 1 \\
		Kickers per Chicane & 16 & 2 \\
		Power of Kicker Amplifiers & 500~kW & 20~kW \\
		Angular Deflection per Kicker & \(\pm94~\mathrm{\mu rad}\) & 
		\(\pm560~\mathrm{\mu rad}\) \\
		Correction Range & \(\pm 10^\circ\) & \(\pm 6^\circ\) \\
		Correction Bandwidth & \(>17.5\)~MHz & \(>23\)~MHz \\
		Phase Monitor Resolution & \(\ll 0.20^\circ\) & \(0.12^\circ\)  \\
		Initial Phase Jitter & \(2.0^\circ\) & \(\sim 0.9^\circ\) \\
		Corrected Phase Jitter & \(0.2^\circ\) & \(0.2^\circ\) \\
	\end{tabular}
\end{ruledtabular}
\end{table}


%\subsection{\label{ss:hardware}Hardware}

%The PFF system uses three phase monitors, two electromagnetic kickers, kicker 
%amplifiers and a digitiser/feedforward controller.

The three phase monitors used at CTF3 \cite{phMonEuCard} are designed and 
constructed by INFN Frascati, with the associated electronics built by CERN. 
The phase monitors are cylindrical cavities with an aperture of 23~mm and a 
length of 19~cm. 
Notch filters, small ridges, in the cavity create a resonating 
volume at 12~GHz, whilst also reflecting stray fields.
The fields induced by the beam traversing the cavity contain a position 
independent monopole mode and a position dependent dipole mode. The induced 
fields are extracted in an opposing pair of feedthroughs on the top and bottom 
of the cavity. The unwanted position dependence is removed by summing the 
output from each feedthrough in a hybrid. 
To extract the phase dependence of the beam signal the output from the hybrids 
is mixed with a 12~GHz reference signal, derived from a 3~GHz source 
time-locked to CTF3 and common to all three phase monitors.
In the electronics for each phase monitor the beam and reference signals are 
split between eight separate mixers, with the output from each combined to give 
the final phase dependent outputs. This has allowed a resolution of 
\(0.12^\circ\), or about 30~fs, to be achieved whilst maintaining linearity 
between \(\pm70^\circ\) [REF]. The quoted resolution 
is determined by comparing the measurements of the two adjacent upstream 
monitors (see Fig.~\ref{fig:pffLayout}).

The kicker amplifiers \cite{RobertsThesis} have been designed and constructed 
by the John Adams Institute/Oxford University. They have a modular design, 
consisting of a central control module, and two drive and terminator modules 
(one per kicker). The control module distributes power and input signals to the 
drive modules. The 20~kW drive modules consist of low voltage Si FETs driving 
high voltage SiC FETs, and for an input voltage of \(\pm2\)~V give an output of 
up to \(\pm700\)~V. The output is linear within 3\% for input voltages between 
\(\pm1.2\)~V, and has a bandwidth of 47~MHz for small signal variations up to 
20\% max output. For larger signal variations the bandwidth is slew rate 
limited. After being applied to the kicker strips, the drive voltage returns to 
the amplifier, where it is monitored and terminated on the terminator modules.

The two electromagnetic stripline kickers \cite{kickerIPAC11} were also 
designed and built by INFN Frascati,  and are based on the DAFNE design [REF]. 
Each kicker is approximately 1~m in length, and has an internal diameter of 
40~mm between the two strips placed along the horizontal walls of the structure.
The kickers are designed to give a fast response of a few ns to the input 
signal, and to give high kick efficiency. The strips have tapered tapered ends 
to reduce beam coupling impedance.
A voltage of 700~V, the maximum output of the amplifiers, applied to the 
downstream ends of the kicker strips yields a horizontal deflection of 1~mrad 
for the 135~MeV CTF3 beam.

Finally, the Feedforward digitiser and controller (FONT5a board) 
\cite{RobertsThesis} was also 
designed and built by John Adams Institute/Oxford University. This digitises 
the processed phase monitor signals and then calculates and outputs the 
appropriate voltage with which to drive the amplifier in order to correct the 
phase. The board consists of a Virtex-5 field programmable gate array (FPGA), 
nine 14-bit analogue to digital converters (ADCs) clocked at 357~MHz, and four 
digital to analogue converters (DACs). The parameters of the correction, such 
as the system timing and gain, are controlled on the board via a LabVIEW data 
acquisition and control system. The FONT5a board also controls the correction 
timing. The combined hardware latency for the PFF system is approximately 
100~ns, and the output from the FONT5a board is delayed by an additional 30~ns 
so that the drive voltage from the amplifiers and the beam arrive at the 
kickers synchronously. %(so the total system latency including 250~ns cable 
%transit times is equal to the 380~ns beam time of flight between the upstream 
%phase monitor and the first kicker in the chicane).

%\subsection{\label{ss:optics}Chicane Optics}

As well as the hardware challenges, the PFF system places additional 
constraints on the optics of the correction 
chicane, and also on the beam lines between the upstream phase monitor and the 
chicane. 

These ensure the maximum possible phase shift per volt applied to the 
kickers without degrading the transverse beam orbit and beam size after the 
chicane, and high correlation between the initial (uncorrected) upstream and 
downstream phase.

\textcolor{red}{Mention CLIC chicane is a different shape}

The correction range of the PFF system is defined by the kicker design, the 
maximum output voltage of the kicker amplifiers, and the optics transfer matrix 
coefficient \(R_{52}\) between the kickers in the chicane, which relates the 
change in path length through the chicane per unit 
deflection at the first kicker. 
For the maximum amplifier output of \(\pm700\)~V the kickers deflect the beam 
by \(\pm0.56\)~mrad. Together with \(R_{52} = 0.74\)~m in the chicane optics 
these define the system correction range of approximately 
\(\pm400~\mathrm{\mu m}\), or \(\pm6^\circ\).

The measured phase shift in the chicane versus the amplifier input voltage is 
shown in Fig.~\ref{fig:corrRange}, and agrees with the expected range. 
However, the response of the amplifier and therefore the phase shift is 
non-linear. The correction algorithm assumes linearity, but this has a 
negligible effect compared to the limitations placed by the
upstream-downstream phase correlation and phase monitor resolution.

%\textcolor{red}{MADX units for R52, R56, i.e. conversion between distance and 
%phase.}

\begin{figure}
	\includegraphics[width=\columnwidth]{figs/corrRange}
	\caption{\label{fig:corrRange}Downstream phase vs. the kicker amplifier 
	input voltage. Standard errors on the measured phase are shown.}
\end{figure}

The PFF system also should not change the beam orbit after the chicane. The 
chicane optics are designed so that the second kicker closes the orbit bump 
created by the first kicker.
Fig.~\ref{fig:orbClos} shows the horizontal beam orbit in the region 
of the chicane for the maximum and minimum kick. The closure in 
the BPMs following the chicane is better than 0.1~mm, compared to a maximum 
offset of 1.5~mm inside the chicane.

\begin{figure}
	\includegraphics[width=\columnwidth]{figs/orbClos}
	\caption{\label{fig:orbClos}Horizontal orbit in and around the TL2 chicane 
	at maximum (blue) and 
	minimum (red) 
	input to the kicker amplifiers. Markers show the measured position in beam 
	position monitors, and dashed lines the predicted orbit using the 
	CTF3 MADX model and hardware parameters.}
\end{figure}

%\textcolor{red}{All this must be achieved whilst keeping dispersion low, 
%matching betas etc. 
%within constraints of pre-existing buildings. Achieved R52 0.74m with max 
%dispersion 1.16m...}

\subsection{\label{ss:r56} Phase Correlation}

\textcolor{red}{The PFF system acts to subtract the measured upstream phase 
(\(\phi_u\)) from 
the initial downstream phase (\(\phi_d\)) with a gain factor (\(g\)):
%\begin{equation*}
\(\phi_{\mathrm{PFF}} = \phi_d - g\phi_u\)
%\end{equation*}
, where \(\phi_{\mathrm{PFF}}\) is the corrected downstream phase. The optimal 
system gain is given by:
\(g = \rho_{ud} \sigma_d/\sigma_u\)
, where \(\sigma_u\) and \(\sigma_d\) are the initial upstream and downstream 
phase jitter respectively, and \(\rho_{ud}\) is the correlation between the 
upstream and downstream phase. The theoretical limit on the corrected 
downstream phase jitter (\(\sigma_{\mathrm{PFF}}\)) with this gain is given by:
\(\sigma_{\mathrm{PFF}}=\sigma_d \sqrt{1-\rho_{ud}^2}\).}

One of the key challenges in operating the PFF prototype at CTF3 has been 
obtaining high correlation between the initial, uncorrected, upstream and 
downstream phase. A correlation of 97\% is required to reduce a typical initial 
phase jitter of \(0.8^\circ\) at CTF3 to the target of \(0.2^\circ\). 

The achievable correlation depends on the phase monitor resolution and any 
additional phase jitter introduced in the beam lines between the upstream and 
downstream phase monitors. The phase monitor resolution of \(0.12^\circ\) 
limits the maximum upstream-downstream phase correlation to
\(98\%\) in typical conditions, and places a theoretical limit of
\(0.17^\circ\) on the \textcolor{red}{measured corrected downstream phase 
jitter.}

Any beam jitter that changes the time of flight of bunches influences the 
resulting downstream phase stability and upstream-downstream phase correlation. 
The dominant source of uncorrelated downstream phase jitter at CTF3 is beam 
energy jitter being transformed in to phase jitter in the transfer lines 
between the upstream and downstream phase monitors.

\begin{figure}
	\includegraphics[width=\columnwidth]{figs/r56Scan}% Here is how to import 
	%EPS art
	\caption{\label{fig:r56Scan}Downstream (red points) and upstream (blue 
		points) phase jitter vs. the \(R_{56}\) value in the set TL1 optics. 
		}
\end{figure}

The first order phase-energy dependence can be described via the optics 
transfer matrix coefficient \(R_{56}\):
\(\phi_d = \phi_u + R_{56}(\Delta p / p)\)
, where \(\Delta p / p\) is the relative beam energy offset.
Optimal conditions for the PFF system are obtained when the total \(R_{56}\) 
between the upstream and downstream monitors is zero.
The \(R_{56}\) value in one of the transfer lines at CTF3, TL1, has been tuned 
in order to achieve this, compensating for \(R_{56}\) terms in other beam lines.
Fig.~\ref{fig:r56Scan} shows that with an \(R_{56}\) of around 10~cm in TL1 the 
downstream phase jitter is reduced to the same level as the upstream jitter. 
The upstream-downstream phase correlation is also increased to above 95\%.

However, a large second order phase-energy dependence was also identified and 
this remains uncorrected. This leads to a degradation in upstream-downstream 
phase correlation if there are drifts in beam energy. Energy drifts resulting 
from klystron trips and RF power drifts at CTF3 have made it difficult to 
maintain high phase correlations for timescales longer than 10~minutes as a 
result.

\section{\label{s:results}Results}

\subsection{\label{ss:gScan}Gain Scan}

Fig.~\ref{fig:gScan} shows how the downstream phase jitter depends on the PFF 
system gain. With stable incoming beam conditions the downstream phase jitter 
should depend quadratically on the gain \cite{RobertsThesis}. Taking in to 
account drifts in the initial upstream-downstream phase correlation and 
downstream phase jitter during the scan, which modify the gain--jitter 
relationship, the achieved and predicted performance agree within the error at 
all gains. At CTF3 the optimal system gain is typically in the range 1.0--1.2, 
being larger than unity when there is a small amplification in the downstream 
phase jitter with respect to the upstream phase jitter.

\begin{figure}
\includegraphics[width=\columnwidth]{figs/gScan}% Here is how to import EPS art
\caption{\label{fig:gScan}Downstream phase jitter with the PFF system on at 
different gains. Markers show the measured phase jitter with standard error 
bars. The shaded red region shows the expected performance given the initial 
beam conditions.}
\end{figure}

\subsection{\label{ss:shape}Intra-Pulse Phase Variations}

The PFF correction is shaped to remove phase variations along the 
1.2~\(\mathrm{\mu s}\) CTF3 beam pulse. The predominant intra-pulse feature at 
CTF3
is a roughly parabolic ``phase sag'' of \(40^\circ\) peak-to-peak, resulting 
from the use of RF pulse compression. As this is much larger than the 
\(\pm 6^\circ\) range of the PFF system, only approximately a 400~ns portion of 
the pulse can be optimally corrected. The phase sag would not be present at 
CLIC, where in any case the drive beam pulse length is less than 400~ns.

%2015: (Peak-to-peak variation of 5.76 degrees in initial phase reduced to 
%0.65 degrees in corrected phase -- OR -- standard deviation of phases reduced 
%from 1.68 to 0.26 degrees...).

%2016: Std \(0.960\pm0.003^\circ\) reduced to \(0.285\pm0.004^\circ\) across 
%440~ns portion of pulse. Worse absolute but better removal small features 
%compared to 2015.

\begin{figure}
	\includegraphics[width=\columnwidth]{figs/shape}% Here is how to import EPS 
	%art
	\caption{\label{fig:shape}Effect of the PFF system on intra-pulse phase 
		variations. The pulse shape upstream (green), and downstream with the 
		PFF 
		system off (blue) and on (red) is shown.}
\end{figure}

\begin{figure}
	\includegraphics[width=\columnwidth]{figs/flatness}% Here is how to import 
	%EPS art
	\caption{\label{fig:flatness}Distribution of downstream rms phase values, 
		here 
		referred to as the 
		``pulse flatness'', for each beam pulse with the PFF system off (blue) 
		and 
		on (red).}
\end{figure}

Fig.~\ref{fig:shape} shows the effect of the PFF system on the intra-pulse 
phase variations. \textcolor{red}{The convention at CTF3 is to operate the PFF 
system in 
interleaved mode, with 
the correction applied to alternating pulses only. This allows a measurement of 
the initial (`PFF Off') and corrected (`PFF On') downstream phase to be 
performed concurrently.} The upstream (PFF input) phase is also shown for 
comparison. Vertical dashed lines mark a 440~ns portion of the pulse where the 
correction is optimal, and this range is used to calculate statistics on the 
effect of the system. 

In this range the PFF system flattens the phase, 
and almost all variations are removed. Residual offsets in the phase are still 
present where there are small uncorrelated differences between the shape of the 
initial upstream and downstream phase. Fig.~\ref{fig:flatness} shows the rms 
phase variation within the 440~ns range 
for each beam pulse in the dataset, with the PFF system on and off. The PFF off 
pulses have an rms of \(0.960\pm0.003^\circ\) on average, and this is reduced 
to \(0.285\pm0.004^\circ\) by the PFF system.

The PFF system at CTF3 has been verified to reduce the amplitude of 
phase errors up to a frequency of 25~MHz, exceeding the CLIC requirements.


%Limited by variations in phase propagation along the pulse (energy differences 
%etc.).

\subsection{\label{ss:meanJit}Pulse-to-pulse Jitter}

As well as removing intra-pulse phase variations the PFF system simultaneously 
corrects offsets in the overall mean phase, i.e. any pulse-to-pulse jitter. The 
mean phase of each beam pulse is calculated across the 440~ns range in the 
central portion of the pulse, as shown before in Fig.~\ref{fig:shape}.

\begin{figure}
	\includegraphics[width=\columnwidth]{figs/meanJit}% Here is how to import 
	%EPS 
	%art
	\caption{\label{fig:meanJit}Distribution of the mean downstream phase with 
		the 
		PFF system off (blue) and on (red).}
\end{figure}

Fig.~\ref{fig:meanJit} shows the effect of the PFF system on the pulse-to-pulse 
stability across a dataset around ten minutes in length. An 
initial mean downstream phase jitter of \(0.92\pm0.04^\circ\) is reduced to \(0.20\pm0.01^\circ\) by the PFF 
correction. All correlation between the upstream and downstream jitter is removed, from 
\(96\pm2\%\) to \(0\pm7\%\). The achieved stability is consistent with the theoretical prediction (considering the initial correlation and jitter) of \(0.26\pm0.06^\circ\) within error bars.
%\textcolor{red}{NB: upstream PFF off is \(0.76\pm0.03^\circ\), but 
%\(0.68\pm0.03^\circ\) for 
%PFF on. Helps to explain why achieved is better than predicted.}

This level of stability could not be maintained for longer periods due to 
CTF3's drifting RF sources, eventually leading to degraded 
upstream-downstream phase correlation and phase drifts outside the PFF 
correction range, as previously mentioned. \(0.30^\circ\) phase jitter has been 
achieved in 20~minute datasets. With suitable feedbacks to keep the phase 
within the correction range, and a reduction of the higher order phase-energy 
dependences in the machine optics, the PFF system could achieve CLIC-level 
phase stability continuously.

\textcolor{red}{The PFF system has also been operated 
whilst intentionally varying the incoming mean phase, as shown in 
Fig.~\ref{fig:wiggle}. The PFF system removes the additional phase variations 
and achieves more than a factor 5 reduction in downstream phase jitter, from 
\(1.71\pm0.07^\circ\) to \(0.32\pm0.01^\circ\) in this case.}

\begin{figure}
	\includegraphics[width=\columnwidth]{figs/wiggle}
	\caption{\label{fig:wiggle}Mean downstream phase with the PFF system off 
		(blue) and on (red) vs. time, with additional phase variations added to 
		the 
		incoming phase.}
\end{figure}

%\subsection{\label{ss:pbpJit}Point-by-point Jitter}
%
%\textcolor{red}{I would remove this. Don't think it adds any 
%information beyond mean jitter and correction of shape.}
%
%\begin{figure}
%\includegraphics[width=\columnwidth]{figs/BestFF_pbp}% Here is how to import 
%%%%EPS art
%\caption{\label{fig:BestFF_pbp}Point-by-point jitter.}
%\end{figure}
%
%Point-by-point jitter of x~degrees achieved across a x~ns portion of the 
%pulse, agrees with simulated value...

\section{\label{s:conc}Conclusions}

CLIC requires a PFF system to reduce the drive beam phase jitter by an order of 
magnitude, from \(2.0^\circ\) to \(0.2^\circ\)~at~12~GHz, or better than 50~fs 
stability. A prototype of the system has been 
in operation at the CLIC test facility CTF3, and corrects the beam phase by 
varying the path length through a chicane using two electromagnetic kickers. 

As 
well as the kickers, the system uses newly designed phase monitors with 
\(0.12^\circ\) resolution, high bandwidth 20~kW amplifiers and a low latency 
digitiser/feedforward controller. The system latency, including hardware and 
signal transit times, is less than the 380~ns beam time of flight between the 
input phase monitor and the correction chicane. Therefore, the feedforward 
correction can 
be directly applied to the same bunch initially measured at the monitor.

New optics for the correction chicane and other beam lines at CTF3 have been 
developed to yield the desired phase shifting behaviour and ensure high 
correlation between the initial upstream and downstream phase.

The prototype system has demonstrated \(0.20\pm0.01^\circ\) pulse-to-pulse 
phase jitter on a time scale of ten minutes. It has also been shown to be able 
to flatten intra-pulse phase variations up to a frequency of 25~MHz. On longer 
timescales the performance of the system is limited by changes to the incoming 
beam conditions, in particular beam energy, which would be better controlled in 
any future application at CLIC.

%Drifts, in particular in beam energy, degrade the correlation between the 
%upstream and downstream phase and prevent this level of stability from being 
%demonstrated on longer time scales at CTF3. A key consideration for any future 
%system should be to design beam lines and optics with zero phase-energy 
%dependence, including non-linear dependencies, to solve this issue.

%\textcolor{red}{Try to apply to XFELs/something else.}

\section{\label{s:ack}Acknowledgements}
\begin{acknowledgments}
	We wish to acknowledge everyone involved in the operation of CTF3 for their 
	help and support in realising the PFF system.
\end{acknowledgments}

\bibliography{pff_short}% Produces the bibliography via BibTeX.

\begin{comment}
\newpage
\pagebreak
\section{Some Figures}

\pagebreak
\begin{figure*}[h]
	\includegraphics[width=\textwidth]{figs/ctfpffLayout}% Here is how to 
	%import EPS art
	\caption{}
\end{figure*}

\pagebreak
\begin{figure*}[h]
	\includegraphics[width=\textwidth]{figs/alt/ctfpffLayout_alt}
	\caption{}
\end{figure*}

\pagebreak
\begin{figure*}[h]
	\includegraphics[width=\textwidth]{figs/gScan}% Here is how to import EPS 
	%art
	\caption{}
\end{figure*}

\pagebreak
\begin{figure*}[h]
	\includegraphics[width=\textwidth]{figs/alt/gScan_old}
	\caption{}
\end{figure*}

\pagebreak
\begin{figure*}[h]
	\includegraphics[width=\textwidth]{figs/alt/gScanFull}
	\caption{}
\end{figure*}

\pagebreak
\begin{figure*}[h]
	\includegraphics[width=\textwidth]{figs/alt/stdMeanPhase}
	\caption{}
\end{figure*}

\pagebreak
\begin{figure*}[h]
	\includegraphics[width=\textwidth]{figs/alt/FFTUpDown_20161215_2340}
	\caption{}
\end{figure*}

\pagebreak
\begin{figure*}[h]
	\includegraphics[width=\textwidth]{figs/alt/FFTOnOff_20161215_2340}
	\caption{}
\end{figure*}
\end{comment}

\end{document}
