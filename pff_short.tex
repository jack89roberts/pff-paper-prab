% ****** Start of file apssamp.tex ******
%
%   This file is part of the APS files in the REVTeX 4.1 distribution.
%   Version 4.1r of REVTeX, August 2010
%
%   Copyright (c) 2009, 2010 The American Physical Society.
%
%   See the REVTeX 4 README file for restrictions and more information.
%
% TeX'ing this file requires that you have AMS-LaTeX 2.0 installed
% as well as the rest of the prerequisites for REVTeX 4.1
%
% See the REVTeX 4 README file
% It also requires running BibTeX. The commands are as follows:
%
%  1)  latex apssamp.tex
%  2)  bibtex apssamp
%  3)  latex apssamp.tex
%  4)  latex apssamp.tex
%
\documentclass[%
 reprint,
%superscriptaddress,
%groupedaddress,
%unsortedaddress,
%runinaddress,
%frontmatterverbose, 
%preprint,
%showpacs,preprintnumbers,
%nofootinbib,
%nobibnotes,
%bibnotes,
 amsmath,amssymb,
 aps,
%pra,
%prb,
%rmp,
%prstab,
%prstper,
%floatfix,
]{revtex4-1}

\usepackage{graphicx}% Include figure files
\usepackage{dcolumn}% Align table columns on decimal point
\usepackage{bm}% bold math
%\usepackage{hyperref}% add hypertext capabilities
%\usepackage[mathlines]{lineno}% Enable numbering of text and display math
%\linenumbers\relax % Commence numbering lines

%\usepackage[showframe,%Uncomment any one of the following lines to test 
%%scale=0.7, marginratio={1:1, 2:3}, ignoreall,% default settings
%%text={7in,10in},centering,
%%margin=1.5in,
%%total={6.5in,8.75in}, top=1.2in, left=0.9in, includefoot,
%%height=10in,a5paper,hmargin={3cm,0.8in},
%]{geometry}

\begin{document}

\preprint{APS/123-QED}

\title{Demonstration of 50~fs stability of an \\ electron beam at the CLIC Test Facility CTF3}% Force line breaks with \\
%\thanks{A footnote to the article title}%

\author{J.~Roberts}
 \altaffiliation[Also at ]{JAI, Oxford University.}%Lines break automatically or can be forced with \\
 \email{Jack.Roberts@cern.ch}
\author{R.~Corsini}
\author{P.~Skowronski}%
\affiliation{%
 CERN, Geneva
}%

\collaboration{CTF3 Collaboration}%\noaffiliation

\author{P.~Burrows}
\author{G.~Christian}
\author{C.~Perry}
% \homepage{http://www.Second.institution.edu/~Charlie.Author}
\affiliation{
 John Adams Institute\\
 Oxford University% with \\
}%
%\affiliation{
% Third institution, the second for Charlie Author
%}%
%\author{Delta Author}
%\affiliation{%
% Authors' institution and/or address\\
% This line break forced with \textbackslash\textbackslash
%}%

\collaboration{FONT Group}%\noaffiliation

\date{\today}% It is always \today, today,
             %  but any date may be explicitly specified

\begin{abstract}
Here is the abstract.
%\begin{description}
%\item[Usage]
%Secondary publications and information retrieval purposes.
%\item[PACS numbers]
%May be entered using the \verb+\pacs{#1}+ command.
%\item[Structure]
%You may use the \texttt{description} environment to structure your abstract;
%use the optional argument of the \verb+\item+ command to give the category of each item. 
%\end{description}
\end{abstract}

%\pacs{Valid PACS appear here}% PACS, the Physics and Astronomy
%                             % Classification Scheme.
%\keywords{Suggested keywords}%Use showkeys class option if keyword
                              %display desired
\maketitle

%\tableofcontents

\section{\label{s:intro}Introduction}

CLIC is a proposal for a linear electron positron collider, using a novel two 
beam acceleration concept to achieve a collision energy of up to 3 TeV....

As part of this scheme strict requirements on the stability of the drive beam, including phase stability which must be better than 0.2 degrees at 12~GHz (or about 50 fs)...

All phases quoted in this paper are in degrees at 12~GHz.

Try to apply to XFELs/other applications that need high phase stability if possible...

Expected phase stability with no correction is 2 degrees. Phase feedforward system proposed to reduce incoming phase jitter by an order of magnitude prior to power extraction...

Prototype of the system at the CLIC test facility CTF3...

\section{\label{s:ctfLayout}System Design}

\begin{figure}
	\includegraphics[width=0.5\textwidth]{figs/ctfpffLayout}% Here is how to 
	%import EPS art
	\caption{\label{fig:pffLayout}Schematic of the PFF system at CTF3.}
\end{figure}

A schematic of the PFF system is shown in Fig.~\ref{fig:pffLayout}. The system 
corrects the phase using two electromagnetic kickers installed 
before the first and last dipole in a four bend chicane (in the TL2 transfer 
line). The beam's path length 
through the chicane depends on the magnitude and polarity of the voltage 
applied to the kickers. The phase is measured using a monitor upstream of 
the chicane (in the CT beam line), and then corrected by setting the kicker 
voltage to deflect bunches arriving early at the phase monitor on to longer 
trajectories in the chicane, and bunches arriving late on to shorter 
trajectories. Downstream of the chicane, in the TBL line, another phase monitor 
is placed to measure the effects of the correction.

The beam time of flight between the upstream phase monitor and the first kicker 
in the chicane is 380~ns. By bypassing the combiner ring (CR) and TL1 transfer 
line, see Fig.~\ref{fig:pffLayout}, the total cable length required to 
transport signals between the monitor and kickers is shorter, approximately 
250~ns. The PFF correction in the chicane can therefore be applied to the same 
bunch initially measured at the phase monitor, providing the total system 
hardware latency is less than 130~ns. The system has a bandwidth of around 
30~MHz, able to remove phase variations along the 1.2~\(\mu s\) CTF3 beam 
pulse, as well as any offsets in the overall mean phase.


\subsection{\label{ss:hardware}Hardware}

The PFF system uses three phase monitors, two electromagnetic kickers, kicker 
amplifiers and a digitiser/feedforward controller.

The three phase monitors are designed and built by INFN Frascati, with the 
associated electronics built by CERN. The monitors are 12~GHz resonating 
cavities with a dipole and monopole mode present. The output from opposing 
vertical pairs of feedthroughs are summed in hybrids to create a position 
independent signal. This signal is split and mixed with a reference 12~GHz 
signal in eight separate mixers. The output from the eight mixers is combined, 
allowing a resolution of 0.126 degrees to be 
achieved whilst maintaining good linearity. This resolution is determined by 
comparing the measurements of the two monitors installed in the CT line.

The two electromagnetic kickers were also designed and built by INFN Frascati,  
and are based on the DAFNE design. A voltage of 1.26~kV applied to the 
downstream end of the kicker strips yields a horizontal deflection of 1~mrad 
for the 135~MeV CTF3 beam.

The kicker amplifiers have been designed and built by the John Adams 
Institue/Oxford University. For 
an input voltage of 2~V gives an output of up to 700~V. Response linear within 
3\% for input voltages up to \(1.2\)~V, then starts to saturate. Bandwidth 
47~MHz for small signal variations up to 20\% max output...

Finally, the Feedforward digitiser and controller (FONT5a board) was also 
designed and built by John Adams Institute/Oxford University. This takes the 
processed phase monitor signals then calculates and outputs the appropriate 
voltage with which to drive the amplifier. 9 ADCs, FPGA, 4 DACs... Digitises 
output from phase monitor electronics, calculates amplifier output based on set 
gain values, deals with correction timing...

\subsection{\label{ss:hardware}Optics}

Two kickers are placed before first and last dipole in the chicane. Installed inside wide aperture quadrupoles to maintain functionality of the lattice...

Key figure of merit for the PFF optics is the transfer matrix coefficient R52, which relates the applied kick to the resulting phase shift...

PFF system should not degrade transverse stability of beam after chicane. Can be achieved by requiring R11=-1 and R12=0 between kickers...

All this must be achieved whilst keeping dispersion low, matching betas etc. within constraints of pre-existing buildings. Achieved R52 0.74m with max dispersion 1.16m...

But had to accept R56 of -0.18m. Means introduction of additional energy dependent jitter downstream not present upstream. PFF system (at CTF) requires residual R56 below 1cm to achieve 97\% correlation. Created new optics for TL1 line with varying R56, to control this effect...

\subsection{\label{ss:commiss}Commissioning}

(MIGHT BE BETTER TO MERGE POINTS HERE IN TO OTHER SECTIONS?)

Correction range is 5.5 degrees, consistent with kicker design and amplifier output. Means only a portion of the CTF3 pulse can be corrected due to phase sag (or maybe could avoid this by only including central part of pulse in plots)...

Beam time of flight 380~ns between first phase monitor and first kicker. Overall measured system latency is 350~ns, including all hardware and cables. Correct same bunch originally measured...

(Need to) demonstrate that downstream orbit is independent of applied kick...

\begin{figure}
\includegraphics[width=0.5\textwidth]{figs/orbClos}% Here is how to import EPS art
\caption{\label{fig:orbClos}Orbit closure.}
\end{figure}


\section{\label{s:results}Results}

\subsection{\label{ss:gScan}Gain Scan}

Theoretical limit on the corrected jitter is \(\sigma_d \sqrt{1-\rho_{ud}^2}\)...

Optimal gain value is \(g = \rho \sigma_d/\sigma_u\)...

Would like new data here but would be nice to: have a scan with clear jitter vs. gain relationship (difficult because of propagation drifts). Otherwise, scatter plot like one I've included could be used to demonstrate effect...

\begin{figure}
\includegraphics[width=0.5\textwidth]{figs/gScan}% Here is how to import EPS art
\caption{\label{fig:gScan}Gain scan - NEED NEW DATA.}
\end{figure}


\subsection{\label{ss:meanJit}Mean Jitter}

(General question: With/without wiggling, mean and point-by-point -- probably can't include all 4? Which to focus on? Need to at least quote lowest achieved jitter, but wiggling results can look more impressive in plots)

Achieved jitter on the mean phase of 0.24 degrees...

Consistent with theoretical value of (TO CALCULATE) given initial correlation, jitter in this dataset...

Assuming 0.1 degrees resolution on mean (best achieved), measured jitter corresponds to actual beam jitter of 0.22 degrees...

\begin{figure}
\includegraphics[width=0.5\textwidth]{figs/BestFF_meanJit}% Here is how to import EPS art
\caption{\label{fig:meanJit}Best mean phase jitter.}
\end{figure}


\subsection{\label{ss:shape}Pulse Shape}

High bandwidth correction - not only correcting the mean but also variations along the pulse...

Peak-to-peak variation of 5.76 degrees in initial phase reduced to 0.65 degrees in corrected phase -- OR -- standard deviation of phases reduced from 1.68 to 0.26 degrees...

\begin{figure}
\includegraphics[width=0.5\textwidth]{figs/BestFF_shape}% Here is how to import EPS art
\caption{\label{fig:shape}Correction of pulse shape.}
\end{figure}

\subsection{\label{ss:pbpJit}Point-by-point Jitter}

\begin{figure}
\includegraphics[width=0.5\textwidth]{figs/BestFF_pbp}% Here is how to import EPS art
\caption{\label{fig:BestFF_pbp}Point-by-point jitter.}
\end{figure}


Point-by-point jitter of x~degrees achieved across a x~ns portion of the pulse, agrees with simulated value...

Limited by variations in phase propagation along the pulse (energy differences etc.), plus resolution slightly worse for point by point than for mean.



\section{\label{s:conc}Conclusions}

PFF prototype at CTF3 has demonstrated phase stability close to CLIC requirements...

\bibliography{apssamp}% Produces the bibliography via BibTeX.

\end{document}
%
% ****** End of file apssamp.tex ******
